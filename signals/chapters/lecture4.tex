\section{Properties of $ \mathscr{ FT }  $ on $ \mathscr{ L } ( \mathbb{R}) $ }
\label{sec:Properties of $ \mathscr{ FT }  $ on $ \mathscr{ L } ( \mathbb{R}) $ }
\begin{ftheo}[Exchange Property]
    Assume $ f, g\in \mathscr{L}^1( \mathbb{R})    $
    Then 
    \[
        \int\limits_{-\infty}^{\infty} f(u) \widehat{g}(u) \ du =
        \int\limits_{-\infty}^{\infty} \widehat{f}(u)g(u) \ du 
    \]
    \label{theo:Exchange Property}
\end{ftheo}

\begin{proof}
    $ -f \in \mathscr{L}^1, \widehat{g} \in \mathscr{ C } ^{\infty} \implies f\widehat{g}
    \in \mathscr{L}^1 $ then use fubini's theorem. 
\end{proof}


\subsection{FT and Derivation}
\label{subsec:FT and Derivation}
\begin{enumerate}
    \item Assume $ x^kf(x) $ in $ \mathscr{L}^1\left( \mathbb{R}\right)   $ 
        for $ k\in \set{ 0,\cdots,n }  $. Thus $ \widehat{f}  $ is k times derivable and 
\[
    \widehat{f} _{  }^{ (k) } (\lambda) = \widehat{\left( -2i\pi x\right) ^k f(x) } \left(
    \lambda \right) 
\]
    \item Assume $ f\in \mathscr{ C } ^n $ and $ f', \cdots, f^{(n)} \in
        \mathscr{L}^1\left( \mathbb{R}\right)  $.
    Then 
    \[
        \widehat{f^{(k)}}\left( \lambda \right) = \left( 2i\pi\lambda \right) ^k
        \widehat{f}\left( \lambda\right) 
    \]
    \item $ f\in \mathscr{L}^1\left( \mathbb{R}\right),\ f   $ bounded support then $
        \widehat{f} \in \mathscr{ C } ^{\infty}  $
\end{enumerate}
\begin{proof}
    \begin{enumerate}
        \item Direct application of the theorem of derivation 
            \[
            \int\limits_{ }^{ } \frac{ \partial  }{ \partial \lambda  } f\left( \lambda,
            t\right)  = \frac{ \partial  }{ \partial \lambda  } \int\limits_{ }^{ }
            f(\lambda, t) \ dt
            \]
        \item $ n = 1 $. $ f' \in \mathscr{L}^1\left( \mathbb{R}\right)   $. 
            \[
                \widehat{f'}\left( \lambda\right) = \lim_{a\to\infty}
                \int\limits_{a}^{a} e^{ -2i\pi\lambda t} f'(t) \ dt 
            \] integration by parts 
            \[
                \widehat{f'}\left( \lambda \right) = \lim_{a\to\infty} \left[ e^{
                -2i\pi\lambda t} f(t)  \right] ^a_{-a} + \lim_{a\to\infty}
                \int\limits_{-a}^{a } 2i\pi\lambda e^{ -2i\pi \lambda t } f(t) \ dt       
            \]
            \[
                \lim_{a\to\infty} \left[ e^{ -2i\pi\lambda t} f(t) \right] _{ -a }^{ a }  
            \]
            \[
                \lim_{a\to\infty} f(t) = 0 \quad \text{ since } \quad f\in
                \mathscr{L}^1p\left( \mathbb{R}\right) .  
            \]
            So, 
            \[
                \widehat{f'}\left( \lambda \right) = 2i\pi\lambda \widehat{f}\left(
                \lambda \right) 
            \]

     \item $ f\in \mathscr{L}^1\left( \mathbb{R}\right)  $ and $ f $ has a bounded
         support. Then what can we say about $ t^kf(t)  $. It is also in $
         \mathscr{L}^1\left( \mathbb{R}\right)   $ and using point 1 we get that $ f\in
         \mathscr{ C } ^{\infty}  $
    \end{enumerate}
\end{proof}
\subsection{Notations}
\label{subsec:Notations}
\begin{itemize}
    \item[symmetry] : 
        \[
            f _{ \sigma }^{  } (x) = f(-x) 
        \]
    \item[Translation] : 
        \[
        \mathscr{ T } af(x) = f(x-a) 
        \]
\end{itemize}



\subsection{Properties}
\label{subsec:Properties}
Assume f in L1 and $ \widehat{f}\left( \lambda \right) = \mathscr{ F } f(\lambda)  $. 
\begin{enumerate}
    \item $\overline{ \mathscr{ F } f} = \overline{ \mathscr{ F } } \overline{f} $
    \item $\left( \mathscr{ F } f\right) _{\sigma} = \overline{ \mathscr{ F } } f =
        \mathscr{ F } f_{\sigma} $
    \item $ f $ even $ \implies  \widehat{f} $ even 
    \item $ f $ odd $ \implies  \widehat{f} $ odd 
    \item $ f $ real and even then $ \widehat{f}  $ real and even   
    \item $ \widehat{ \mathscr{ T } a f}\left( \lambda \right) = e^{ -2i\pi\lambda a}
        \widehat{f}(\lambda)  $ time delay
    \item $ \mathscr{ T } a \widehat{f}\left( \lambda\right) = \widehat{ e^{ 2i\pia t}f(t)
        }\left( \lambda \right)   $ frequency delay
\end{enumerate}

\subsection{Usual examples}
\label{subsec:Usual examples}
\begin{align*}
    e^{ -ax} u(x) &= \frac{ 1 }{ a+2i\pi\lambda  }  \\ 
    e^{ ax} u(-t)  &= \frac{ -1 }{ -a + 2i\pi\lambda}  \\ 
    \frac{ x^k }{ k! } e^{ -ax} u(x)  &=  \frac{ 1 }{ \left( a + 2i\pi\lambda \right)
    ^{k+1}  } \\ 
        \frac{ x^k }{ k! } e^{ax} &= \frac{ -1 }{ \left(-a + 2i\pi\lambda\right)^{k+1}  }  \\ 
        e^{ -a \left | x \right | }  &= \frac{ 2a }{ a^2 + 4\pi^2\lambda^2 }  \\ 
        sign(x) e^{ -a \left | x \right | }  &= \frac{ -4i\pi\lambda }{ a^2 +
        4\pi^2\lambda^2 }  \\
            e^{-at^2} /& \sqrt{ \frac{ \pi }{ a } } e^{ \frac{ -\pi^2 }{ a^2 } \lambda^2}    \\
            \chi _{[-a, a]} (t)  &= \frac{ \sin(2a\pi\lambda)  }{ \pi x }  \\ 
\end{align*}

\subsection{Inverse of FT}
\label{subsec:Inverse of FT}
\begin{ftheo}[Inverse of FT]
    If $ f \in \mathscr{L}^1\left( \mathbb{R}\right)   $ and $ \widehat{f} \in
    \mathscr{L}^1\left( \mathbb{R}\right)   $. Then 
    \[
        \overline{ \mathscr{ F } } \widehat{f}(t) = f(t) , \quad \forall t \text{ where f
        cont} 
    \]
    But, $ f\in \mathscr{L}^1\left( \mathbb{R}\right) notimply \widehat{f} 
    \mathscr{L}^1\left( \mathbb{R}\right)   $ 
    \label{th:Inverse of FT}
\end{ftheo}

\begin{cor}[]
    $ f \in \mathscr{L}^1\left( \mathbb{R}\right)   $ and $ f \in \mathscr{ C } ^2  $ and
    $ f, f', f'' \in \mathscr{ L } ^1 $. Then 
    \[
        \widehat{f} \in \mathscr{L}^1\left( \mathbb{R}\right) .  
    \]
\end{cor}

\begin{proof}
    \[
        \widehat{f''}\left( \lamda \right) = -4\pi^2\lambda^2 \widehat{f}(\lambda) 
    \]
    and 
    \[
        \lim_{ \left | \lambda  \right | \to \omega } \left | \widehat{f''} \left( \lambda
        \right)  \right | = 0
    \] 
    Then 
    \[
    \exists M s.t. \left | \lambda  \right | > M \implies 4\pi^2\lambda^2 \left |
    \widehat{f} \left( \lambda \right)  \right | < 1
    \]
    $ \left | \widehat{f} \right | $ is continuous (RL theom). 
    \[
        \text{At } \omega \quad \left | \widehat(\lambda)  \right | < \frac{ 1 }{
        4\pi^2\lambda^2 } 
    \]
    Then 
    \[
        \left | \widehat{f}  \right | \in \mathscr{L}^1\left( \mathbb{R}\right)  
    \]
\end{proof}


\begin{cor}[]
    $ f, \widehat{f} \in \mathscr{L}^1  $, we have 
    \[
        \mathscr{ F } \widehat{f}\left( u\right) = f _{ \sigma  }^{  } (u) = f(-u) 
    \]
\end{cor}

\begin{exmp}[]
    Consider the previous examples. 
    \[
        \frac{ 1 }{ \left( a + 2i\pi t \right) ^{k+1}  } = \frac{ \left( -\lambda\right)
        ^k  }{ k! } e^{ a\lambda } u\left( -\lambda \right) 
    \]
    Also, 
    \[
        \frac{ -1 }{ \left( -a + 2i\pi t \right) ^{k+1}  } = \frac{ \left( -\lambda\right)
        ^k }{ k!  }e^{-a\lambda} u(\lambda)
    \]
    And 
    \[
    \frac{ 1 }{ a^2 + t^2  } = \frac{ \pi }{ a } e^{ -2i\pi \left | \lambda  \right | } 
    \]
\end{exmp}



\section{FT on $ \mathscr{ S } \left( \mathbb{R}\right)  $ }
\label{sec:FT on $ \mathscr{ S } \left( \mathbb{R}\right)  $ }
We need to restrict $ \mathscr{L}^1\left( \mathbb{R}\right)   $ in order to 
\begin{itemize}
  \item inverse FT
  \item Use derivation formulas 
\end{itemize}
In order to do this we use the Schwartz space. 
\begin{exmp}[Schwartz space ]
    \[
    \mathscr{ S } \left( \mathbb{R}\right) \subset \mathscr{L}^1\left( \mathbb{R}\right)  
    \]. 
\end{exmp}

\begin{defn}[Rapidly Decreasing Functions]
    \[
        \forall p \in \mathbb{N}, \lim_{p \to \infty} \left | x^p f(x)  \right | = 0
    \]
    \label{def:Rapidly Decreasing Functions}
\end{defn}

\begin{ftheo}[Property 1]    
If $ f \in \mathscr{L}^1\left( \mathbb{R}\right)  $ and is a rapidly decreasing function,
then 
\begin{enumerate}
    \item $ \forall p, t^pf(t) \in \mathscr{L}^1\left( \mathbb{R}\right)   $
    \item $ \widehat{f} \in \mathscr{ C } ^{\infty}  $
\end{enumerate}
    \label{th:}
\end{ftheo}
\begin{proof}
    $ f $ is rapidly decreasing. 
    \[
        \implies \forall p \in \mathbb{N} \quad \lim_{ \left | x \right | \to \infty}
        \left | x^{p+2} f(x)  \right | = 0
    \]
    \[
        \imples \exists M s.t. \left | x \right | > M  \quad \left | x^{p+2} f(x)  \right
        | < 1   
    \]
    \[
    \int\limits_{ }^{ } \left | x^p f(x)  \right | \ dx = \int\limits_{ \left | x \right |
    < M }^{ } \left | x^p f(x)  \right | \ dx + \int\limits_{ \left | x \right | > M
}^{}  + \int\limits_{ \left | x \right | > M }^{ } \frac{ 1 }{ x^2  } \left | x^{p+2} f(t)
\right | \ dt 
    \]
    \[
        \leq M^p \int\limits_{ \left | x \right | < M }^{ } \left | f(x)  \right | \ dx +
        \int\limits_{ \left | x  \right | > M }^{ } \frac{ 1  }{ x^2  } \ dx
    \]
    \[
    \leq \infty 
    \]
    Then $ \forall p, \quad x^p f(x) \in \mathscr{L}^1\left( \mathbb{R}\right)   $. 
    Then
    \[
        \widehat{ \left( -2i\pi t \right) ^k f(t) } 
    \] exisits and it is $ \widehat{f}^{(k)} \left( \lambda \right)  $

\end{proof}

\begin{ftheo}[Property 2]
    Let $ f \in \mathscr{ C } ^{\infty}  $ and $ \forall k  $ $ f^{(k)} \in
    \mathscr{L}^1\left( \mathbb{R}\right)   $. Then $ \widehat{f}  $ is rapidly
    decreasing.     
    \label{th:Property 2}
\end{ftheo}
\begin{proof}
    \[
        \widehat{f _{  }^{ (k) } }\left( \lambda \right) = \left( 2i\pi\lambda \right) ^k
        \widehat{f} \left( \lambda \right)          
    \]
    By RL theorem 
    \[
        \lim_{ \left | \lambda  \right | \to \infty} \left | \widehat{f _{  }^{ (k)  } }
         (\lambda)  \right| = 0  
    \]
    Then 
    \[
        \lim_{ \left | \lambda  \right | \to 0} \left | \lambda^k \widehat{f}(\lambda)
        \right | = 0
    \]
\end{proof}
\subsubsection{Conclusion }
The more regular $ f  $ is, the more rapidly $ \widehat{f}  $ converges to zero at
infinity. The more rapidly $ f $ converges, the more regular $ \widehat{f} $ is. 
Furthermore, we realize that if $ f \in \mathscr{ C } ^{\infty}  $ and $ f  $ rapidly
decreasing, then $ \widehat{f}  $ rapidly decreasing and $ \widehat{f} \in \mathscr{ C }
^{\infty}   $. 

\begin{defn}[Schwartz Space]
    The vectorial space $ \mathscr{ S } \left( \mathbb{R}\right)  $ of function are those
    function for which 
    \[
     f \in \mathscr{ C } _{  }^{ \infty }  
    \]
    \[
     f _{  }^{ (k)  } \text{ are rapidly decreasing for all k}  
    \]
    \label{def:Schwartz Space}
\end{defn}

We have for $ \mathscr{ S } \left( \mathbb{R}\right) \subset \mathscr{L}^1\left( \mathbb{R}\right)   $
\begin{enumerate}
    \item $ \mathscr{ S }  $ stable for derivatives 
    \item $ \mathscr{ S }  $ stable for $ \mathscr{ FT }  $
    \item $ \mathscr{ S }  $ stable for x by a polynomial 
\end{enumerate}

$ \mathscr{ FT }  $ is a bijection from $ \mathscr{ S } \left( \mathbb{R}\right)  $ to
itself. 

\chapter{FT on $ \mathscr{L}^2(\mathbb{R} )    $}
In signal processing, we often work with 
\[
\int\limits_{ }^{ } \left | f(t)  \right | ^2 \ dt = \text{ the energy of the signal} =
\|f\|_{2} 
\]
$ \mathscr{L}^2(\mathbb{R})  $ is the "natural" space for signals. To define $ \mathscr{
FT }  $ on $ \mathscr{L}^2(\mathbb{R})  $, we use the density of $ \mathscr{ S } \left(
\mathbb{R}\right)  $ in $ \mathscr{L}^2(\mathbb{R})  $.
\begin{ftheo}[Extension Theorem]
    $ E, F $ vectorial spaces with norm. F is complete. Let G be a dense subspace of E.
    Let A be a linear continuous application from G to F. Then $ \exists ! $ extension of
    $ A : E \to F  = \overline{A} $. $ \overline{A} $ is lin and cont and $ \|
    \overline{A} \|^{ }_{ E, F} = \| A \|^{ }_{ G, F}  $. With 
    \[
    \| A \|^{ }_{ G, F} = Sup \left( \frac{ \| A f \|^{ }_{ F}  }{ \| f \|^{ }_{ G}  } \right) 
    \]
    \label{th:Extension Theorem}
\end{ftheo}


\section{Properties of $ \mathscr{ FT }  $ on $ \mathscr{L}^2(\mathbb{R})  $}
\label{sec:Properties of $ \mathscr{ FT }  $ on $ \mathscr{L}^2(\mathbb{R})  $}
\begin{enumerate}
    \item $ \forall f\in \mathscr{L}^2(\mathbb{R})  $. $ \mathscr{ F\overline{F} } f =
        \mathscr{ FF } f = f $ almost everywhere.
    \item $ \forall f,g \in \mathscr{L}^2(\mathbb{R}) \times \mathscr{L}^2(\mathbb{R})  $. 
        \[
            \int\limits_{ }^{ } f(t) \overline{g}(t) \ dt = \int\limits_{ }^{ } \mathscr{
            F} f(\lambda) \overline{ \mathscr{ F } g} \left( \lambda \right) \ d\lambda 
        \]
        \[
            \langle f,g \rangle _{ L^2 }^{  } = \langle \widehat{f} , \widehat{g}  \rangle
            _{ L^2 }^{  } 
        \]
    \item 
        \[
            \| f \|^{ }_{ 2} = \| \widehat{f} \|^{ }_{ 2} 
        \]
        energy conservation or Parseval equality. 
    \item Exchange property. $ f,g \in \mathscr{L}^2(\mathbb{R}) \times \mathscr{L}^2(\mathbb{R})  $
        \[
            \int\limits_{ }^{ } f(u) \widehat{g}(u) \ du = \int\limits_{ }^{ }
            \widehat{f}(u) g(u) \ du 
        \]
\end{enumerate}
\begin{exmp}[]
    We can do IFT on the functions listed previously to see if we can perform the IFT. 
\end{exmp}

\section{FT and Convolution}
\label{sec:FT and Convolution}
\subsubsection{Part A}

$ f,g \in \mathscr{L}^2\left( \mathbb{R}\right)   $. 
Then 
\begin{enumerate}
    \item $\widehat{f*g} \left( \lambda \right) = \widehat{f}(\lambda) \cdot
        \widehat{g}(\lambda)  $
    \item $ \widehat{f \cdot g} \left( \lambda \right) = \widehat{f} * \widehat{g} \left(
        \lambda \right)  $
\end{enumerate}

\subsubsection{Part B}
$ f, g \in \mathscr{ S } \left( \mathbb{R}\right)  $, then 
\begin{enumerate}
    \item $\widehat{f*g} = \widehat{f}\cdot\widehat{g} $
    \item $\widehat{f\cdot g} = \widehat{f} * \widehat{g}$
\end{enumerate}

\subsubsection{Part C}
$ f,g \in \mathscr{L}^2(\mathbb{R})  $
\begin{enumerate}
    \item $f * g(t) = \overline{F}\left( \widehat{f} \widehat{g}\right) (t) $
    \item $ \widehat{f\cdot g} = \widehat{f} * \widehat{g} (\lambda) $
\end{enumerate}


\section{Heisenberg Uncertainty Principle}
\label{sec:Heisenberg Uncertainty Principle}
\subsection{Exercises}
\label{subsec:Exercises}

\subsection{Interpretation of Heisenberg's Uncertainty Principle}
\label{subsec:Interpretation of Heisenberg's Uncertainty Principle}
Let $ f(t)  $ be our signal and $ \widehat{f}(\lambda) $ be its frequential
representation. We say that 
\[
    \int\limits_{ }^{ } \left | f(t) \right | \ dt = \int\limits_{ }^{ } \left |
    \widehat{f}(\lambda) \right |^2 \ dt
\] is the energy of the signal and $ \left | f(t)  \right | ^2 $ is the local density. 
\begin{itemize}
  \item We can view a signal as its density of energy. 
  \item We can summarize $ f(t)  $ by the box there is between time and frequency
      accuracy. 
      \[
          \mu_N : \sigma_t\sigma_{\epsilon} > \frac{ 1 }{ 4\pi } 
      \]
\end{itemize}

\begin{figure}[ht]
    \centering
    \incfig{heisunbox}
    \caption{heisUnBox}
    \label{fig:heisunbox}
\end{figure}

\section{Conclusion About Fourier Transform}
\label{sec:Conclusion About Fourier Transform}
Very useful because  : 
\begin{enumerate}
    \item Transform Convolution into product
    \item Transforms derivative into polynomial
    \item Easy to compute using the FFT algorithm
\end{enumerate}
But, this is a tool adopted for the use of "regular" signals, stationary signals. The tool
is not performed for iregular sizes. Example : 
Show $ \delta_0  $ and its fourier Transform. 

Consider 
\[
    f(t) = \left( \sin(2\pi 5t) + \sin(2\pi 20t) \right) \mathbbm{1}_{[0,1[} \text{ and } 
    g(t) = \sin(2\pi 5t) \mathbbm{1}_{[0,\alpha[} + \sin(2\pi 20t)\mathbbm{1}_{[\alpha,1[} 
\]

Use graphs from python to explain! 

To correct these drawbacks. We use 
\begin{itemize}
  \item Window Fourier Transform
  \item Gabor Transform
  \item Continuous Wavelet Transform and Discrete Wavelet Transform
\end{itemize}



