\chapter{Fourier Series}
\section{Trigonometric Polynomials}
\label{sec:Trigonometric Polynomials}
We begin by creating a space of functions, to this end we will need to define the
functions that constitute this space and their properties. The basis of this space is
given by monocromatic functions.
\begin{defn}[Monocromatic Function]
    \[
        D \circ f : e_n(t) = e^ \frac{ 2i\pi nt  }{ T }   
    \]
    $ e_n(t) $ is the monochromatic function.     \label{def:Monocromatic Function}
\end{defn}

\begin{figure}[ht]
    \centering
    \incfig{monocromatic}
    \caption{monocromatic}
    \label{fig:monocromatic}
\end{figure}
\newpage
We say that there is one frequency, thus one
    chroma. Since n is fixed the frequency is constant. The n subscript references the
    frequency, thus for greater n there is a greater frequency.   

\begin{defn}[Periodic Function]
    A funcion $ f $ is called periodic with period $ a $ if 
    \[
    \forall t\in \mathbb{R} \quad f(t+a) = f(t) 
\]
    \label{def:Periodic Function}
\end{defn}
The function $ e_n(t) $ has period $ T $ for all $ n \in \mathbb{N} $, thus, we can write
any function $ p $ using the following : 
\begin{defn}[Trigonometric Polynom]
    We define a trigonometric polynomial as : 
    \[
        p(t) = \sum_{n=-N}^{N} c_ne_n(t), \qquad c_n \in \mathbb{R} 
    \]
    which has degree less that or equal to $ N$. 
    Expanding this representation we get : 
    \[
        p(t) = c_0 + \sum_{n=1}^{N} \left( C_ne^ \frac{ 2i\pi nt }{ T } +  C_{-n}e^ \frac{
        2i\pi nt }{ T }\right) 
    \] 
    using the fact that $ \cos $ is even and $ \sin  $ is odd we get 
    \[
        p(t) = C_0 + \sum_{n=1}^{N} \left( \left( C_n + C_{-n}\right)\cos \left( \frac{ 2\pi nt
            }{ T } \right) +
        i\left( C_n - C_{-n}\right)\sin\left( \frac{ 2\pi nt }{ T}\right) \right)   
    \]
Which we abbreviate to 
\begin{equation}
        p(t) = C_0 + \sum_{n=1}^{N} \left( a_n\cos\left( \frac{ 2\pi nt }{ T } \right) +
        b_n\sin\left( \frac{ 2\pi nt }{ T } \right) \right)  
        \label{eq:fourier_decomp}
\end{equation}
    where 
    \[
        a_n = C_n + C_{-n} \quad b_n = i\left( C_n - C_{-n}\right) 
    \]
    which gives us the following inverse formulas : 
    \[
        c_n = \left( a_n - ib_n)\right) / 2 \quad c_{-n} = \left( a_n + ib_n \right) / 2
    \]
    \label{def:Trigonometric Polynom}
\end{defn}

\subsection{Orthogonality}
\label{subsec:Orthogonality}

We can calculate 
\begin{align*}
\langle e_n , e_m \rangle &= \int\limits_{0}^{T} e_n(t) \bar{e_m}(t) \\ 
                          &= \int\limits_{0}^{T} e^ { 2i\pi(n - m)  /  T }   \\ 
  &= \begin{cases}
      0 &\text{ if } n \neq m \\
      T &\text{ if } n = m \\ 
  \end{cases}
\end{align*}
We use 
$ \mathscr{ T }  _N $ to denote the set of trigonometric polynomials of order N. It is a vectorial space
generated by $ \set{e_n(t)}_{-N \leq n \leq N} $
$ \\ $
The functions 
$ e_n  $ and $ e_m  $ are orthogonal in $ \mathscr{L}^2\left( T\right)  $, using the scalar product
on $
\mathscr{L}^2(T)$. Where T denotes the period and thus the interval where the functions are defined.  
\[
    \langle f , g \rangle _{L^2} = \int\limits_{0}^{T} f(t) \overline{g(t)} \ dt 
\] and \[
\| e_n \|^{2 }_{ } = \langle e_n , e_n \rangle =  T
\] and 
\[
    \| e_n \|^{ }_{ } = \sqrt{T}
\]

The orthonormal basis of $ \mathscr{  T}_N$ is given by $ \set{ \frac{ e_n(t) }{ \sqrt{T} }}
$ For $ p(t) \in \mathscr{T}_N  $.

\subsection{Calculating Fourier Coefficients}
\label{subsec:Calculating Fourier Coefficients}
\section{Need to check these!!!}
\label{sec:Need to check these!!!}
\subsubsection{For Period = $ 2\pi $}
Using \cite{asmar_fourpde} Section 2.2 we derive the formulas for coefficients that are in
\ref{eq:fourier_decomp}, however, for simplicity we derive this for functions with period
$ 2\pi $. 
Integrate both sides of \ref{eq:fourier_decomp} except we do for x instead of t. Also, the
period is $ 2\pi $ so the equation becomes 
\[
\int\limits_{-\pi }^{\pi} f(x) \ dx = \int\limits_{-\pi}^{\pi} a_0 \ dx +
\sum_{n=1}^{\infty} \int\limits_{-\pi}^{\pi} \left( a_n \cos nx + b_n \sin nx\right) \ dx
\] Since 
\[
\int\limits_{-\pi}^{\pi} \cos nx \ dx = \int\limits_{-\pi}^{\pi} \sin nx \ dx = 0 \qquad
\text{ for } n = 1, 2, 3, \cdots    
\]
we have 
\[
    \int\limits_{-\pi}^{\pi} f(x) \ dx = \int\limits_{-\pi}^{\pi} a_0 \ dx = 2\pi a_0
\] or 
\[
a_0 = \frac{ 1 }{ 2\pi } \int\limits_{-\pi}^{\pi} f(x) \ dx
\]

We use orthogonality to derive the other terms. 
\[
    \int\limits_{-\pi}^{\pi} f(x) \cos mx \ dx =  \overbrace{ \int\limits_{-\pi}^{\pi} a_0
    \cos mx \ dx }^{= 0} + \sum_{n=1}^{\infty} \overbrace{ \int\limits_{-\pi}^{\pi} a_n
\cos nx \cos mx \ dx }^{= 0 \text{ for } m \neq n} + \sum_{n=1}^{\infty} 
\overbrace{ \int\limits_{-\pi}^{\pi} b_n\sin nx \cos mx \ dx  }^{=0}
\]
\[
    = a_m  \overbrace{\int\limits_{-\pi}^{\pi} \cos^2 mx \ dx  }^{=\pi} = \pi a_m
\]
Thus, 
\[
a_m = \frac{ 1 }{ \pi } \int\limits_{-\pi}^{\pi} f(x) \cos mx \ dx \quad \left( m = 1, 2,
\cdots \right) 
\]
Similarly, we reach 
\[
b_m = \frac{ 1 }{ \pi } \int\limits_{-\pi}^{\pi} f(x) \sin mx \ dx \quad \left( m = 1, 2,
\cdots \right) 
\]
\subsubsection{Period = T}
We now use \ref{eq:fourier_decomp} as it is, with period T. 

\[
    \int\limits_{0}^{T}    p(t)  = \int\limits_{0}^{T}  C_0 + \sum_{n=1}^{N}
     \int\limits_{0}^{T} \left( a_n\cos\left( \frac{ 2\pi nt }{ T } \right) +
        b_n\sin\left( \frac{ 2\pi nt }{ T } \right) \right)  
\]
The rightmost expression is always zero for $ n = (1, 2, \cdots) $. Thus, 
\[
\int\limits_{0}^{T} p(t) \ dt = \int\limits_{0}^{T} C_0 \ dt = TC_0
\]
Gives us 
\[
   \frac{ 1 }{ T }   \int\limits_{0}^{T} p(t) \ dt = C_0
\]
Using orthogonality we can compute cosine representation : 
\begin{align*}    
    \int\limits_{0}^{T} p(t) \cos\left( \frac{2\pi mt  }{ T } \right) \ dx &=
    \overbrace{\int\limits_{0}^{T} C_0 \cos\left(\frac{ 2\pi mt }{ T }  \right) \
    dt}^{=0}\\ 
     &+\sum_{n=1}^{\infty}   \overbrace{\int\limits_{0}^{T} a_n \cos\left(\frac{ 2\pi nt }{ T }  \right)
     \cos\left( \frac{ 2\pi mt }{ T } \right)  \ dt}^{=0 \text{ for } m\neq n}  
  \\ &+ \sum_{n=1}^{\infty}  \overbrace{\int\limits_{0}^{T} C_0 \sin\left(\frac{ 2\pi nt }{ T
     }\cos\left( \frac{ 2\pi mt }{ T } \right)   \right) \ dt}^{=0}\\
     &= \int\limits_{0}^{T} a_n\cos^2\left( \frac{ 2\pi mt }{ T } \right) \ dt \\ 
     &= \left[ \frac{ t }{ 2 } + \frac{ T }{ 8\pi m } \sin\left( \frac{ 4\pi m t }{ T }
     \right) \right] ^T_0 \left( b/c \int\limits_{ }^{ } \cos^2 ax \ dx = \frac{ x }{ 2 }
 + \frac{ 1 }{ 4a } \sin 2ax + C\right) \\ 
     &= \frac{ Ta_n }{ 2 }  \\ 
\end{align*}
which gives us 
\[
a_n = \frac{ 2 }{ T } \int\limits_{0}^{T} p(t) \cos\left( \frac{2\pi mt  }{ T } \right) \
dt 
\]
Similarly, 
\[
    b_n = \frac{ 2 }{ T } \int\limits_{0}^{T} p(t) \sin\left( \frac{ 2\pi nt }{ T }
    \right) \ dt
\]
Using 
\[
    C_n = \int\limits_{-T/2}^{T/2} p(t) e_n(t) \ dt 
\]
we observe that if $ p(t) $ is even then $ C_n = C_{-n} \implies b_n = 0 $ and if $ p(t) $
is odd then $ C_n = - C_{-n} \implies a_n = 0 $. 

\begin{ftheo}[Perseval equality]
    \[
        \| p(t) \|^{ 2}_{ L^2(T)} = \langle \sum_{n=-N}^{N} C_ne_n(t) , \sum_{m=-N}^{N}
        C_mb_m(t) \rangle 
    \]
    this gives us 
    \[
    T \sum_{n=-N}^{N} \left | C_n \right | ^2 \iff \sum_{}^{} \left | C_n \right | ^2 =
    \frac{ 1 }{ T } \int\limits_{0}^{T} \left | p(t) \right | ^2 \ dt
    \]
    \label{th:Perseval equality}
\end{ftheo}
\begin{ftheo}[Parseval's Equation (alternate form)]
    $ \\ $Let 
    \[
        f(x) = \sum_{k = -\infty}^{\infty} \alpha_k e_n(t) \in \mathscr{ L^2 }_{(T)} 
    \]
    Then 
    \[
    \frac{ 1 }{ T } \| f \|^{ 2}_{ } = \frac{ 1 }{ T } \int\limits_{0}^{T} \left | f(x)
    \right | ^2 \ dx = \sum_{k=-\infty}^{\infty} \left | \alpha_k \right | ^2
    \]
    Moreover, for f and g in $ \mathscr{L}^2[T] $ we obtain 
    \[
    \frac{ 1 }{ T } \langle f , g \rangle = \frac{ 1 }{ T } \int\limits_{0}^{T} f(t)
    \overline{g(t) } \ dt = \sum_{n=\infty}^{\infty} \alpha_n \overline{\beta_n}
    \]
    \label{th:Parseval's Equation (alternate form)}
\end{ftheo}
A physical interpretation of this equation is that the energy of a signal is simply the
sum of the energies from each of its frequency components. 




\section{Fourier Series}
\label{sec:Fourier Series}
Let's consider $ \mathscr{ L }^2(T) = \set{f, \text{T-periodic} , \text{ and } \int\limits_{0}^{T}
\left | f(t) \right | ^2 dt < \infty} \\$
On $ \mathscr{L}^2(T) $ we use the scalar product 
\[
    \langle f , g \rangle = \int\limits_{0}^{T} f(t) \bar{g}(t) \ dt \text{ and } \| f
    \|^{ 2}_{ } = \langle f , f \rangle 
\]
$ \mathscr{ L }^2(T) $ is a Hilbert Space. $ \\ $
Remark : What does it mean to say that $ \| f \|^{ }_{ } = 0 ? $ this means that 
the integral of the function is zero almost everywhere. 
\[
\int\limits_{0}^{T} \left | f(t)  \right | ^2 \ dt = 0 
\] all equalities that we deal with are not necessarily true for each point of the two
functions. Thus $ f $ and $ g $ can be said to be equal almost everywhere even if $ f $ is
continuous everywhere and $ g $ is irregular. 

$ \\ $

We have $ \mathscr{ T } _N < \mathscr{ L }^2(T)  $ thus, $ \mathscr{ T } _N $ is a
vectorial space included in $ \mathscr{ L } ^2(T)  $. In fact, we will project elements of $
\mathscr{ L } ^2(T) \text{ on }  \mathscr{ T } _N $! View this projection as an approximation of any
function of $ \mathscr{ L } ^2(T)  $ by a trigonometric polynom.

$ \\ $
We want, by definition $ \forall p \in \mathscr{ T } _N, \ p  $ orthogonal $ \left( f -
f_N\right) \implies \forall |n| \geq N , \ e_n $ orthogonal $ \left( f - f_N\right)  $ and 
\[
\langle e_n , f \rangle = \langle e_n , f_N \rangle 
\]

For $ f_N \in \mathscr{ T } _N $ we can write $ f_N $ as
\[
    f_N(T) = \sum_{n=-N}^{N} C_n e_n(t) 
\] then we have $ \langle f_N , e_N \rangle = TC_n $ so 
\begin{defn}[Fourier coefficient]
    \[
        C_n = \frac{ 1 }{ T } \langle f , e_n  \rangle 
    \]
    $ C_n $ is the docomposition factor of $ e_n $ of the projection $ f $ on $ \mathscr{
    T} _N $. 
    \label{def:Fourier coefficient}
\end{defn}

\begin{defn}[Using Pythagorous]
    \[ \| f \|^{ 2}_{ }  = \| f_N \|^{ 2}_{ }  + \| f - f_N \|^{ 2}_{ } \]
    \[
    \| f \|^{ 2}_{ } = T \sum_{n = -N}^{N} \left | C_n \right | ^2 + \| f - f_N \|^{ 2}_{ } 
    \]
\end{defn}
which gives Bessel inequality : 
\begin{defn}[Bessel inequality]
    \[
    \sum_{-N}^{N} \left | C_n \right | ^2 \leq \frac{ 1 }{ T } \| f \|^{ 2}_{ } \implies
    \sum_{n = -N}^{N} \left | C_n \right | ^2 \text{ is convergent } 
    \]
    Thus, $ C_n \to 0 $ as $ n \to \infty  $
    
    \label{def:?}
    \label{def:Bessel inequality}
\end{defn}
\begin{ftheo}[Theorem of Convergence in $ \mathscr{ L } ^2 $]
    Let $ f \in \mathscr{ L } ^2(T) $. 
    \[
        f_N(t) = \sum_{n=-N}^{N} C_n e^ \frac{ 2i]pi nt }{ T } 
    \]
    \[
        f_N(t) \to f(t) \quad \text{ as } N \to \infty 
    \] 
    This means that 
    \[
        \int\limits_{0}^{T} \left | f_N(t) - f(t)  \right | ^2 \ dt \to 0 \text{ as } N \to
        \infty 
    \]
    it does imply the pointwise convergence.        
    \label{th:Theorem of Convergence in $ \mathscr{ L } ^2 $}
\end{ftheo}
\begin{proof}
    The proof is based on a theorem of Hilbert spaces. 
    \begin{ftheo}[]
        Let $ \left( \varphi_n\right) _n $ an orthonormal basis of H and $ \left( \alpha _
        n \right) _n $ scalars. 
        Then 
        \[
        \sum_{}^{} \alpha_n \varphi_n \text{ converges } \iff \sum_{}^{} \left | \alpha_n
        \right | ^2 \| \varphi_n \|^{ 2}_{ H} < \infty 
        \]
        \label{th:}
    \end{ftheo}
    Assume $ \sum_{}^{} \alpha_n \varphi_n < \infty$. By the continuity of the norm 
    \[
    \| \sum_{}^{} \alpha_n \varphi_n  \|^{ 2}_{ } < \infty \implies \sum_{}^{} \left |
    \alpha \right | ^2 \| \varphi _n \|^{ 2}_{ }  < \infty. 
    \]
    The other way : 
    We assume $ \sum_{}^{} \left | \alpha_n \right | ^2 \| [hi_n \|^{ 2}_{ } < \infty $ we
    can show that $ \sum_{}^{} \alpha_n \varphi_n $ is a Cauchy sequence : .... missed
    half the proof. 
    \[
    p < q, \ \| \sum_{ \left | n \right | < p }^{}  \|^{ }_{ } 
    \]
\end{proof}


\begin{ftheo}[Perseval equality]
    \[
    T \sum_{}^{} \left | C_n \right | ^2 = \| f \|^{ 2}_{ } - \| f - f_N \|^{ ^2}_{ } \to
    0 \quad N \to \infty 
    \]
    \[
    \sum_{n=-w}^{w} \left | C_n \right | ^2 = \frac{ 1 }{ T } \| f \|^{ 2}_{ } 
    \]
    \[
        \sum_{n=-w}^{\infty} C_n(g)\bar{C_n(g)} = \frac{ 1 }{ T } \int\limits_{0}^{T} f(t)
        \bar{g}(t) \ dt
    \]
    \label{th:Perseval equality}
\end{ftheo}

Properties of the Fourier Coefficients. 
\begin{itemize}
  \item Unicity 
      $ f = g $ almost surely then $ \forall n,\ C_n(f) = C_n(g) $
      $ f = 0 $ almost everywhere then $ \forall n \ C_n(f) = 0 $. 
  \item f real function, then $ C_n = \bar{C_n}  $, $ a_n, b_n real $
  \item $ f $ even then $ C_n  $ even and $ b_n = 0 $. 
  \item $ f  $ odd then $ C_n  $ odd and  $ \left( a_n = 0\right)  $
    \item $ f $ even and real $ \implies  $ $ \left( C_n\right)  $ is even and a real
        sequence. 
    \item $ f $ odd and real $ \implies  $ $ \left( C_n\right)   $ is a odd and imaginary
        sequence. 
\end{itemize}

\section{Pointwise representation of periodic function by its Fourier Series}
\label{sec:Pointwise representation of periodic function by its Fourier Series}
In practice, we don't know $ f $ for $ t \in [0,T] $ in fact, in practice, $ f $ is
represented by a vector if $ \mathbb{R}^N $ which is $ \left( f(t_0) , f(t_1), \cdots ,
f(t_{N-1}) \right)  $ using this vector which we denote by $ ?f(t)  $ we approximate the
fourier coefficients. So we need to be able to inverse the Fourier Series and obtain the
intial signal. So we need to have
pointwise convergence for this to be achievable. 

To have it, we need to ave some regularity properties of $ f$. 

\begin{ftheo}[]
    Assume $ f $ is T periodic, continuous, and differentiable, almost everywhere. 
    Assume that $ f' $ is piecewise continuous, Then : 
    $ \\ $
    \begin{enumerate}
        \item The fourier series of $ f' $ is obtained from Fourier Series of $ f $
            differentiating each term. 
        \item Fourier Coefficients of $ \left( f\right)  $ satisfy 
            \[
            \sum_{-\infty}^{\infty} \left | C_n \right | < \infty  
            \]
        \item The Fourier Series of $ f $ converge uniformly to $ f $.  
    \end{enumerate}
    \label{th:diff_convergence_FS}
\end{ftheo}

\begin{proof}
    \begin{enumerate}
        \item Use integration by parts on 
            \[
                C_n(f) = \left[ \frac{ T }{ 2i\pi n } f(t)e^ {-\frac{2i\pi nt  }{ T }}
                \right]^T_0 + \frac{ 1 }{ T } \frac{ T }{ 2i\pi n } \int\limits_{0 }^{T}
                f'(t) e^ {- \frac{ 2i\pi nt }{ T } } \ dt 
            \] 
            \[
                = \frac{ 1 }{ 2i\pi n  } \int\limits_{0}^{T} f'(t)e^{- \frac{ 2i\pi nt }{
                T} } \ dt  
            \]    
            which is $ TC_n(f')  $
            and 
            \[
                C_n(f') = \frac{ 2i\pi n  }{ t } C_n(f)
            \]
        \item
            \[
                \left | C_n(f)  \right | = \frac{ T }{ 2\pi n  } \left | C_n\left( f'\right)  \right | 
            \]
            \[
                \leq \frac{ T }{ 4\pi  } \left( \frac{ 1 }{ n^2 } + \left | C_n(f')
                \right |^2 \right) 
            \]

        \item Normal convergence of $ \left | C_n \right | \implies \sum_{}^{} \left | C_n
            \right | < \infty \implies f_N $ normal convergence to $ g $. 
            \[
                f_N \to f \text{ in } \mathscr{ L } ^2(t) \implies f_N \text{ uniformly
                convergent to } g.
            \]
            \begin{align*}
                \text{ if }  &f_N \to f \text{ in L2 }   \\
                \text{ if }  &f_N \to f \text{ in ? }   \\
            \end{align*} $ \implies  f = g$ almost everywhere. 
            $ f $ is continuous then $ f_N \to f $ uniformly. 
    \end{enumerate}
\end{proof}

\begin{ftheo}[Regularity Theorem]
    $ f $ is T periodic. $ f \in \mathscr{ C }^p \implies \exists k \in \mathbb{R}  $ s.t. 
    \[
        \left | C_n(f) \right | \leq \frac{ k }{ \left | n  \right | ^p } 
    \]
    \label{th:Regularity Theorem}
\end{ftheo}
Conclusion : 
The more regular f is, the more rapidly $ C_n(f) $ converges to 0. 
\begin{enumerate}
    \item $ f \in \mathscr{ L } ^1(T) \implies C_n(f) \to 0 $
    \item $ f \in \mathscr{ L } ^2(T) \implies \sum_{}^{} \left | C_n(f) \right | ^2 <
        \infty $
    \item $ f \in \mathscr{ L } ^1(T) \implies \sum_{}^{} \left | C_n(f) \right | <
        \infty $
    \item $ f \in \mathscr{ C } ^2(T) \implies \left | C_n(f) \right | \leq \frac{ k }{
        n^2 }  $ for some $ k \in \mathbb{R}  $. 
    \item $ f \in \mathscr{ C } ^\infty(T) \implies \forall k \in \mathbb{N}  $ s.t. $ \left |
        n^kC_n(f)\right | \to 0 $ 
\end{enumerate}


From the exercises with the characteristic function, we see that $\forall n, \ f_n(0) = 0$ 
Fourier series only capture the continuous or regular parts of a funtion. It does not
capture the discontinutities or irregularities of the function. Look to exercise 5 to see
that $ f = g $ even though g has an impluse of 2 at $ \pi /2 $. 
$ \\ $Irregularities are lost in the Fourier Series representation.

\section{Exercises 1 : Fourier Series}
\label{sec:Exercises 1 : Fourier Series}
\subsection{Speed of convergence of Fourier coefficients and functional regularity}
\label{subsec:Speed of convergence of Fourier coefficients and functional regularity}
\begin{enumerate}
    \item Let $ f\in \mathscr{ C } ^2_p(a) $, prove that $ \left | C_n(f) \right | \leq
        \frac{ k }{ n^2 }  $
    \item Let $ f\in \mathscr{ C } ^\infty _P(a) $, prove that 
        \[
            \forall k \in \mathbb{N}, \lim_{ \left | n  \right | \to \infty} \left | n^k
            c_n(f) \right | = 0
        \]
\end{enumerate}

\subsection{Fourier series convergence}
\label{subsec:Fourier series convergence}
Let $ f \in \mathscr{L}^2(2\pi) $ defined by : 
\[
    f(t) = \chi_{[0, \pi[} (t) - \chi_{[\pi, 2\pi[} (t)
\]
where $ \chi_{[a,b]} $ is the characteristic function on the interval $ [a,b] $. Note $
f_N(t) $ the Fourier series of order N. 
\[
    f_N(t) = \sum_{n=-N}^{N} c_n(f) e^{2i\pi n t / a}
\]

\begin{enumerate}
    \item What does it mean to say that $ f_N  $ converges to $ f $ in $ \mathscr{ L }
        ^2_P(2\pi) $. 
    \item Calculate $ f_1(t), f_2(t), f_3(t), f_5(t) $
    \item What are the values of $ f(t) , f_1(t) , f_2(t) , f_3(t) , f_5(t)  $ for $ t =
        \pi $
    \item Does $ f_N(t) $ pointwise converge to $ f(t) $? 
    \item Let the function $ g\in \mathscr{L}^2_P(2\pi) $ be defined as :
        \[
            g(t) = \chi_{[0, \pi / 2[} (t) - \chi_{]\pi/2, \pi]} (t) + 2\delta_{\pi /2}(t) - 
            \chi_{[\pi, 2\pi[} (t)
        \]
        Note $ g_N(t) $ its Fourier transform of order N. Calculate $ g_1(t) $. What can
        you conclude. 
\end{enumerate}
Solutions to \ref{subsec:Speed of convergence of Fourier coefficients and functional
regularity} $ \\ $ 
1) 
\begin{proof}
    From the proofs given in \ref{sec:Pointwise representation of periodic function by
    its Fourier Series} we know that 
    \[
        \left | C_n(f) \right | = \frac{ T }{ 2i\pi n }  \left | C_n(f') \right | 
    \]
    Thus, 
    \[
        \left | C_n(f') \right | = \frac{ T }{ 2i\pi n } \left | C_n(f'') \right | 
    \]
    which gives 
    \begin{align*}
        \left | C_n(f) \right | &= \frac{ T }{ 2i\pi n } \left( \frac{ T }{ 2\pi n  } \left
        | C_n(f'') \right | \right)  \\
                                &= \frac{ T^2 }{ 4\pi^2 n^2 } \left | C_n(f'') \right |  \\ 
    \end{align*}
    We can use this to show that for arbitrary $ \epsilon >0 $ we have that there exists
    an N such that $ n \geq N $ satisfies  
    \[
        \left | C_n(f) \right | = \frac{ T }{ 2i\pi n } \left | C_n(f') \right | = \frac{
        T^2}{ 4\pi^2n } \left | C_n(f'') \right | < \epsilon  
    \]
    Then 
    \[
        \left | C_n(f) \right | \leq \epsilon \frac{ T^2 }{ 4\pi^2n^2 } = \frac{ k }{ n^2
        } \qquad \text{ for } k = \epsilon  \frac{ T^2  }{  4\pi^2} 
    \]

\end{proof}
2) 
\begin{proof}
    Since $ f \in \mathscr{ C }^\infty _P(a) $ and \ref{th:diff_convergence_FS} (2) we
    have 
    \begin{align*}
        \left | C_n(f) \right | = \frac{ T }{ 2i\pi n  } \left | C_n(f') \right | = \cdots 
        = \left( \frac{ T }{ 2\pi n  } \right) ^q \left | C_n(f^{(q)}) \right |   
    \end{align*}
   for arbitrary q. Thus, for q =  2k we have 
   \[
       \left | n^k C_n(f) \right | = \frac{ T^{2k}n^k }{ 2^{2k}\pi^{2k}n^{2k}  } \left |
       C_n(f^{(2k)} \right | 
   \] 
   since 
   \[
       \left | n^k C_n(f)  \right | = n^k \left | C_n(f) \right | 
   \]
   we have 
   \[
       \left | C_n(f) \right | = \frac{ T^{2k} }{ 2^{2k}\pi^{2k}n^{k}  } \left |
       C_n(f^{(2k)} \right |  
   \]
   which goes to zero as $ n \to \infty $ Note : (Could have just used $ k+1 $ rather than
   $ 2k $.) 
\end{proof}

Solutions to \ref{subsec:Fourier series convergence}. 
$ \\ $
1) It means that it converges on average to $ f $. There may be points where this isn't
true. This will be show in the 5th exercise. 

$ \\ $
2) 
We first consider $ f(t) $. 

\begin{figure}[ht]
    \centering
    \incfig{exercisesheet1}
    \caption{f(t)}
    \label{fig:exercisesheet1}
\end{figure}
\newpage

Which is a real odd function, thus, $ C_n $ is odd and imaginary sequence and $ a_n = 0 $. 
So we need only calculate the $ b_n $ in order to find the coefficients and we need only
consider odd numbers.  
\begin{align*}
    f_n(t) &= \frac{ 2 }{ \pi  } \int\limits_{0}^{2\pi} \sin \left( \frac{ 2\pi nt }{ 2\pi
    } \right) \ dt \\   
    f_1(t) &=  \frac{ 2 }{ 2\pi  } \int\limits_{0}^{2\pi} \sin \left( nt \right) \ dt \\ 
           &= \int\limits_{0}^{\pi } \sin(t)\ dt - \int\limits_{\pi}^{2\pi} \sin(t) \ dt \\ 
    &= -\frac{ 1 }{ \pi } \left[ \cos(t) \right] _{ 0 }^{ \pi } + \frac{ 1 }{ \pi }
     \left[ \cos(t)\right] _{ \pi }^{ 2\pi } \\
      &= -\frac{ 1 }{ \pi  } \left( 1 - \left( -1\right) \right) + \frac{ 1 }{ \pi }
      \left( -1 - 1 \right) = - \frac{ 4 }{ \pi  }  \\ 
     &\text{ we can generalize this to all n } \geq 1  \\ 
    & -\frac{ 1 }{ \pi } \left[ \frac{ 1 }{ n }  \cos(nt) \right] _{ 0 }^{ \pi } + \frac{ 1 }{ \pi }
     \left[ \frac{ 1 }{ n }  \cos(nt)\right] _{ \pi }^{ 2\pi } \\
      &\text{ Since n is always odd we have }  \\
      &= - \frac{ 1 }{ \pi n  } \left[ \left( 1 - \left( -1\right) \right) - \left( -1 - 1
      \right) \right]  = -\frac{ 4 }{ \pi n }   \\ 
\end{align*}
Thus, $ b_n = - \frac{ 4 }{ \pi n }  $ and we can calculate the coefficients $ C_n, C_{-n}
$ using the identities, where $ a_n = 0 $, thus  
\[
    C_n = -ib_n / 2 \qquad C_{-n} = ib_n / 2
\]
\[
    C_1 = -ib_1 / 2 = \frac{ -i\left( -\frac{ 4 }{ \pi }  \right)  }{ 2 } =
    \frac{ 2i}{\pi}  
\]
Furthermore, since $ f $ is odd we have $ C_n = - C_{-n}  $ and 
\[
    C_{-1} = - \frac{ 2i }{ \pi  } 
\]
In general, this gives us 
\[
    C_n = \frac{ 4i }{ 2n\pi } \quad \text{ and } \quad C_{-n} = -\frac{ 4i }{ 2n\pi }  
\]
We can finally calculate the functions : 
\[
    F_N(t) = \sum_{n = -N}^{N} C_n(f)e _{  }^{ \frac{ 2i\pi nt  }{ 2\pi }  } 
\]
which yeilds the simplified form : 
\[
    F_N(t) = \sum_{n=-N}^{N} C_n\left( f\right) e _{  }^{ int } 
\]
\begin{align*}
    f_1(t)  &= \sum_{n=-1}^{1} C_n \left( f\right) e _{  }^{ int  }  \\ 
            &= - \frac{ 2i }{ \pi } e _{  }^{ -it } + \frac{ 2i }{ \pi  } e _{  }^{ it }    \\
             &= \frac{ 2i }{ \pi } \left( e _{  }^{ it } - e _{  }^{ -it } \right)  \\ 
             &= \frac{ 2i }{ \pi  } \left( \cos t + i\sin t - \cos(-t) -i\sin(-t)\right)  \\
             &= \frac{ 2i }{ \pi  }\left( 2i\sin(t) \right)  \\
              &= -\frac{4  }{ \pi  } \sin t \\ 
\end{align*}
\begin{align*}
    f_2(t)  &=  \sum_{n=-2}^{2} C_n(f) e^{int} \\
     &= - \frac{ i }{ \pi } e _{  }^{ -2it } - \frac{ 2i }{ \pi } e _{  }^{ -it } + 
     \frac{ 2i }{ \pi  } e _{  }^{ it } + \frac{ i }{ \pi } e^{2it} \\ 
     &= - \frac{ 4 }{ \pi } \left( \sin t + \frac{ 1 }{ 3 }  \sin 3t\right)  \\ 
\end{align*}
\begin{align*}
    f_3(t)  &=  \sum_{n=-3}^{3} C_n(f) e^{int} \\
     &= - \frac{ 4 }{ \pi  } \left( \sin t + \frac{ 1 }{ 3 } \sin 3t + \frac{ 1 }{ 5 }\sin 5t\right)  \\ 
\end{align*}

3) For $ t = \pi  $ we have 
\begin{align*}
    f_1(\pi) &= - \frac{ 4 }{ \pi } \sin(\pi) = 0 \\
    f_3(\pi) &= - \frac{ 4 }{ \pi } \left( \sin(\pi) + \frac{ 1 }{ 3 }\sin(3\pi) \right) =
    0\\ 
    f_5(\pi) &=  - \frac{ 4 }{ \pi  } \left( \sin \pi + \frac{ 1 }{ 3 } \sin 3\pi 
    + \frac{ 1 }{ 5 }\sin 5\pi \right) = 0\\ 
\end{align*}

4) 
It does not converge pointwise to $ f(t) $, consider that $ f_N(\pi) = 0, \ \forall N \in
\mathbb{N}  $  and $ f(\pi) = -1 $. 

5)
\begin{figure}[ht]
    \centering
    \incfig{exercisesheet1p5}
    \caption{Dirac on Characteristic Function}
    \label{fig:exercisesheet1p5}
\end{figure}

Calculate $ g_1(t) $. $ \\ $ 
First, we consider $ C_n(f) $. This has the same Fourier series representation as $ f(t)$.
We can conlude that Fourier series representations capture the continuous properties or
aspects of a function. 


