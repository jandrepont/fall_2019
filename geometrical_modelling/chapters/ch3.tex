\chapter{Curves in the Plane}
\section{Introduction}
\label{sec:Introduction}

To represent them, we use 
\begin{defn}[Parametrized Curves]
    \begin{align*}
        \gamma : [a,b] &\to \mathbb{R}^d \\
        t &\to \gamma(t)  
    \end{align*}
    \label{def:Parametrized Curves}
\end{defn}

\begin{figure}[ht]
    \centering
    \incfig{planar-curve}
    \caption{Parametrized Planar Curve}
    \label{fig:planar-curve}
\end{figure}


\begin{defn}[Implicit Curves]
    Let $ f : \mathbb{R}^2 \to \mathbb{R} $
    then $ f^{-1} ( \set{ 0 } ) = \set{ (x,y) \in \mathbb{R}^2, f(x,y) = 0 }  $
    is a curve.
    \label{def:Implicit Curves}
\end{defn}

\begin{exmp}[]
    \[
        f(x,y) = x^2 + y^2 -1 
    \]
\end{exmp}

\begin{figure}[ht]
    \centering
    \incfig{example313}
    \caption{Example 3.1.3}
    \label{fig:example313}
\end{figure}


\begin{defn}[Graphs of Functions]
    $ \varphi : [a,b] \to \mathbb{R} $. 
    \[
        \text{graph} \left( \varphi\right) = \set{ \left( t, f(t) \right) , t \in [a,b]  } 
    \]
    \label{def:Graphs of Functions}
\end{defn}

\newpage
\begin{exmp}[]
    $ \varphi = \sqrt{1 - t^2}  $ and $ t \in [-1,1]  $ then 
\end{exmp}

\begin{figure}[ht]
    \centering
    \incfig{example315}
    \caption{Example 3.1.5}
    \label{fig:example315}
\end{figure}





\section{Generalities on Paramaterized Curves}
\label{sec:Generalities on Paramaterized Curves}
\subsection{Reminder}
\label{subsec:Reminder}
Let $ f:[a,b] \to \mathbb{R}^2\in \mathscr{ C } ^n $ and $ t_0 \in [a,b] $ then 
\[
    f(t) = f(t_0) + \left( t-t_0\right) f'(t_0) + \frac{ \left( t-t_0\right) ^2 }{ 2!  }
    f''(t_0) + \dots + \frac{ \left( t-t_0\right) ^n }{ n! } f^{(n)} (t_0) + \mathcal{ O
    } \left( \left( t-t_0\right) ^n\right) 
\]

In particular, 
\[
    \frac{ f(t) - f(t_0)  }{ t - t_0  } = f'(t_0) 
\]
let curve be $ \mathscr{ C } = \set{ f(t) }  $ and $ f(t_0) \in \mathscr{ C }  $ and $
f'(t_0)  $ is a vector tangent to $ \mathscr{ C }  $ at $ f(t_0)  $. 

Some notes about 2nd derivative and how it "attracts" the curve. 


\[
    f(t) = f(t_0) + f^{(p)}(t_0) \frac{ \left( t-t_0\right) ^p }{ p! } + \dots + 
    f^{(q)}(t_0) \frac{ \left( t-t_0\right) ^q }{ q! } + \mathcal{ O  } \left( \dots\right) 
\]
p is the smallest k such that $ f^{(k)}(t_0) \neq 0 $ and q is smallest q such that 
\[
    \left( f^{(q)}(t_0),f^{(k)}(t_0)\right)   \text{ independent } 
\]
\newpage
\begin{figure}[ht]
    \centering
    \incfig{linearinependenceoftan}
    \caption{linear independence between derivatives}
    \label{fig:linearinependenceoftan}
\end{figure}

Certain characteristics of the curve can be given by the values of $ p,q $. 


\begin{figure}[ht]
    \centering
    \incfig{pqcurvecharacteristics}
    \caption{pqCurveCharacteristics}
    \label{fig:pqcurvecharacteristics}
\end{figure}

These values then blah blah blah
$ \\ $
\newpage


\section{Parametrization and Geometric Curves}
\label{sec:Parametrization and Geometric Curves}
\begin{defn}[Parametrized Curve]
    A Paramterized curve of class $ \mathscr{ C } ^k $ is a map $ f: I \subset
    \mathbb{R}\to \mathbb{R}^3 \in \mathscr{ C } ^k$, where I is a union of intervals. We
    denote $ \left( I, f\right)  $ such a curve.
    \label{def:Parametrized Curve}
\end{defn }

Remark : 
\[
    F(I) \coloneqq \mathscr{ C } 
\]
is the geometric support. Interval connected $ \implies \mathscr{ C }  $ is connected. 
I compact set $ \implies \mathscr{ C }  $ compact set. 
$ \\ $
Remark : 
$ \\ $
Some curve may have 2 paramterization without the same regularity. 
\begin{exmp}[]
    \begin{align*}
        t &\to \left( t, t^{3/2}\right) t > 0  \\ 
        t &\to \left( \left | t \right | , - \sqrt{t^3}\right) t \leq 0  \\ 
    \end{align*}
    another parametrization is 
    \[
        t \to \left( t^2, t^3\right) \in \mathscr{ C } ^{\infty} 
    \]
\end{exmp}
 

\newpage 
\subsection{ReParametrization}
\label{subsec:ReParametrization}
Let $ f : I \to \mathbb{R}^3 $ param curve in $ \mathscr{ C } ^k $ and $ e : J \to I $ is
a $ \mathscr{ C } ^k $ diffeomorphism (bijective, $ e'(x) \neq 0, \mathscr{ C } ^k $. 
Then $ f \circ e : J \to \mathbb{R}^3 $ has the same "geometric curve" and we say that 
\begin{itemize}
  \item $ f\circ e $ is a reparametrization of $ f $
  \item $ e $ is called an admissible change of variable
\end{itemize}


\begin{figure}[ht]
    \centering
    \incfig{reparamaterized-curve}
    \caption{Reparamaterized Curve}
    \label{fig:reparamaterized-curve}
\end{figure}


We consider the following equivalence class 
\begin{defn}[Equivalence Class for Curves]
    $ \\ $
    $ \left( I,f\right) \sim \left( J,g\right)  $ if 
    $
    \exists e : J \to I,  g = f \circ e  \text{ e admissible change of variable} 
    $
    \label{def:}
\end{defn}

\begin{defn}[Geometric Curve]
    A geometric curve is an equivalence class of this relation
    \label{def:Geometric Curve}
\end{defn}

\section{Regular Curve}
\label{sec:Regular Curve}

\begin{defn}[Regular Curve]
    Let $ k \geq 1 $. 
    We say that a paramatrized curve $ \left( f,I\right)  $ of class $ \mathscr{ C } ^k $
    is regular if 
    \[
        f'(t) \neq 0 \ \forall t \in I
    \]
    A geometric curve is regular if there exists a paramatrized which is regular. 
    \label{def:Regular Curve}
\end{defn}

If $ \mathscr{ C } $ is of class $ \mathscr{ C } ^1 $, then there exists $ f: I \to
\mathscr{ C }  $, where $ \mathscr{ C } = f\left( I\right)  $ 
\[
    f'(t) \neq 0 \quad f'(t) \text{ is tangent to } \mathscr{ C } ^k \text{ at } f(t)
\]

If $ \left( I,f\right)  $ is regular then every reparametrization $ \left( J,g \right)  $ 
is also regular. Indeed : $ \forall t, \ f'(t) \neq 0  $ gives 
\[
    g = f \circ e \implies \forall t,  \ g'(t) = \underbrace{f'(e(t))}_{\neq 0} \times 
    \underbrace{e'(t)}_{\neq 0}  \neq 0
\]

\begin{exmp}[]
    A line segment in $ \mathbb{R}^2 $ with $ t \to \left( t, at+b\right)  $ regular,
    can also be
    reparametrizated by $ t\to \left( t^3, at^3+b\right)  $ non regular. The reason for this is 
    \[
        f'(t) = \left( 3t^2, 3at^2\right) = 0 \text{ at } x = 0
    \]
\end{exmp}
\subsubsection{Remark}
This curve does not admit a regular paramatrization. 

    
\begin{figure}[ht]
    \centering
    \incfig{regular-and-non-regular-curves}
    \caption{Regular and Non-regular curves}
    \label{fig:regular-and-non-regular-curves}
\end{figure}

However, this is not to say that there does not exist parametrizations of these figures,
it is just to say that $ f'(a) = 0 $ where a is the non-smooth point. 

Furthermore, we have 


\begin{figure}[ht]
    \centering
    \incfig{c1-but-not-c2}
    \caption{ $ \mathscr{ C } ^1 $ but not $ \mathscr{ C } ^2 $}
    \label{fig:c1-but-not-c2}
\end{figure}

This curve is $ \mathscr{ C } ^1 $ since $ f''(x) = 0 $. 
\newpage 

\section{Metric Properties of Curves}
\label{sec:Metric Properties of Curves}
\subsection{Length of curves}
\label{subsec:Length of curves}
\begin{defn}[Length of a curve]
    Let $ f : I = [a,b] \to \mathbb{R}^d $. We see that the straight line segments 
    obviously have less length than $ \mathscr{ C }  $. $ \\ $
    Let $ \mathscr{ S }  = \set{ \text{ subdivisions } a=t_0 < \dots < t_n = b } $.

    For $ s \in \mathscr{ S }   $ we denote 
    \[
        \gamma(s) = \sum_{i=0}^{N-1} \| f(t_{i+1}) - f(t_i) \|^{ }_{ }  
    \]If $ \set{ \gamma(s), s \in \mathscr{ S }   }  $ is bounded we say that $ f $ is
    rectifiable. Its lengh is defined by 
    \[
        \gamma(f) = \sup_{s \in \mathscr{ S } } \gamma(s)
    \]
    Then If $ f:[a,b] \to \mathbb{R}^d $ is $ \mathscr{ C } ^1 $ then $ f $ is rectifiable
    and 
    \[
        \gamma(f) = \int\limits_{a}^{b} \| f'(t) \|^{ }_{ } \ dt
    \]
    \label{def:Length of a curve}
\end{defn}
Sketch of proof $ \\ $

\begin{align*}
    \int\limits_{a}^{b} \| f'(t)  \|^{ }_{ } \ dt &= \sum_{i=1}^{n-1}
    \int\limits_{t_i}^{t_{i+1}} \| f'(t)  \|^{ }_{ } \ dt \\ 
\end{align*}
The right hand side of this term gives us 
\begin{align*}
     &\sim \left( t_{i+1} - t_i \right) \| f'(t) \|^{ }_{ }   \\ 
     &\sim \left( t_{i+1} - t_i \right) \frac{ \| f(t_{i+1} - f(t_i)  )  \|^{ }_{ }  }{
     \left | t_{i+1} - t_i \right |  }   \\ 
\end{align*}

\begin{figure}[ht]
    \centering
    \incfig{length-of-a-curve}
    \caption{Length of a Curve}
    \label{fig:length-of-a-curve}
\end{figure}





\subsection{Arc Length Parametrization}
\label{subsec:Arc Length Parametrization}
\begin{defn}[]
    Let $ \left( I,f\right) \in \mathscr{ C } ^1 $ where $ I = [a,b] ,\ t_0 \in I $. We
    call arc-length the map 
    \begin{align*}
        \simga: I &\to \mathbb{R}  \\
        t &\to \int\limits_{t_0}^{t} \| f'(\mu) \|^{ }_{ } \ d\mu \\ 
    \end{align*} 
    \label{def:}
\end{defn}
\subsubsection{Remark}
$ \left | \sigma(t) \right |  $ is the length of the curve between $ f(t)  $ and $ f(t_0)
$ where $ \sigma(t) < 0 \iff t<t_0 $
\subsubsection{Remark}
If $ f $ is regular then $ \sigma  $ is strictly increasing and $ \sigma'(t) = \| f'(t)
\|^{ }_{ } > 0 $, therefore, $ \sigma  $ is an admissable change of variable of class $
\mathscr{ C } ^1 $

\begin{defn}[]
    $ f\circ \sigma^{-1} $ is an arc-length paramatrization of the curve. 
    \label{def:}
\end{defn}
So every $ \mathscr{ C } ^1 $ regular curve admits an arc-length paramatrization.
We use, by convention, $ S $ as the parameter of the arc-length parametrization.

\begin{figure}[ht]
    \centering
    \incfig{arc-length-paramatrization}
    \caption{Arc-length Paramatrization}
    \label{fig:arc-length-paramatrization}
\end{figure}








\begin{prop}[]
    \begin{itemize}
      \item  The arc-length paramatrization is unique up to the parameter $ t_0 $ if $ I = [a,b],\
    t_0 = a $.
\item $ \forall s \in J, \ \| f'(s) \|^{ }_{ } = 1 $ if $ f $ arc-length param 
\item $ \forall s \in J, f'(s) \perp f''(s)  $ for $ f $ arc-length
    \end{itemize}  
\end{prop}
\begin{proof}
    Let $ g $ be any parametrization and $ f = g \circ \sigma^{-1}  $. Then 
    \[
        \forall s \ f'(s) = g'\left( \sigma^{-1} (s) \right) \times \left(
        \sigma^{-1}\right) '(s) = \frac{ g'\left( \sigma^{-1}(s)\right)  }{ \| g'\left(
    \sigma^{-1}(s)\right)  \|^{ }_{ }  } 
    \]
    where 
    \[ \left( \sigma^{-1}\right)' (s) = \frac{ 1 }{ \sigma'\left(\sigma^{-1}(s)\right) } = 
    \frac{ 1 }{ \| g'\left( \sigma^{-1}(s)\right)  \|^{ }_{ }  } \]
    Then $ \| f'(s) \|^{ }_{ } = 1 $ and 
    \[
        \forall s \in J \ \| f'(s) \|^{ 2}_{ } = \langle f'(s) , f'(s) \rangle = 1
    \]
    We derive, $ \forall s \in J  $
    \[
        2 \langle f''(s)  , f'(s)  \rangle = 0 \implies f''(s) \perp f'(s)
    \]
\end{proof}

\section{Planar Curves}
\label{sec:Planar Curves}
\subsection{Serret-Fresnet Frame}
\label{subsec:Serret-Fresnet Frame}
Let $ f : I \to \mathbb{R}^2, \in \mathscr{ C } ^1 $-regular. Then 
\[
    T(t) = \frac{ f'(t)  }{ \| f'(t) \|^{ }_{ }  } \quad N(t) = \text{rot}_{ \frac{ \pi }{
    2}} \left( T(t)\right) 
\]
So $ \left( f(t), \boldsymbol{T}(t), \boldsymbol{N} (t) \right)  $ is a frame that is
called the Serret-Fresnet Frame.

\subsubsection{Remark}
If arc-length then we have 
\[
    T(s) = f'(s) \qquad N(s) = \text{rot} _{ \frac{ \pi }{ 2 } }\left( T(s)\right) 
\]

\subsection{Curvature}
\label{subsec:Curvature}
\begin{defn}[Curvature]
    Let $ f : I \to \mathbb{R}^2 $ arc-length. The curvature at $ f(s)  $ is defined by 
    \[
        k(s) \coloneqq \langle f''(s)  , N(s)  \rangle = \pm \| f''(s) \|^{ }_{ } 
    \]
    \label{def:Curvature}
\end{defn}
\begin{prop}[]
    Let $ f : I \to \mathbb{R} $ any parametrization. Then 
    \[
        k(u) = \frac{ \text{det} \left( f'(u), f''(u) \right)  }{ \| f'(u) \|^{3 }_{ }  } 
    \]
    \label{def:}
\end{prop}

\begin{proof}
    We denote $ \overline{f} = f \circ \sigma^{-1}  $ the arc-length parametrization. We
    put $ u = \sigma^{-1}(s)  $. Then 
    \begin{align*}
        \overline{f}'(s) &= f'\left( \sigma^{-1}(s) \right) \frac{ 1 }{ \| f'\left( \sigma^{-1}
        (s)\right)  \|^{ }_{ }  }  \\
                     &= \frac{ f'(u) }{ \| f'(u) \|^{ }_{ }  }  \\ 
    \end{align*}
    We derive again 
    \begin{align*}
        \overline{f}''(s)  &= \frac{ f''(u) }{ \| f'(u) \|^{2 }_{ }  } + f'(u) \frac{ d }{
        ds}  \\ 
    \end{align*}
    where $ \frac{ d }{ ds }  $ is the real value result of the rhs above 
    \begin{align*}
        \text{det} \left( \overline{f}'(s), \overline{f}''(s) \right)  &= \text{det}
        \left( \frac{ f'(u) }{ \| f'(u) \|^{ }_{ }} , \frac{ f''(u) }{ \| f'(u) \|^{2 }_{
            }}
        + \lambda u f'(u)   \right)  \\ 
                                                                       &=
                                                                       \frac{\text{det}
                                                                       \left(f'(u), f''(u)
                                                                   \right)  }{ \| f'(u)
                                                               \|^{3 }_{ }   }  \\ 
    \end{align*}
    And 
    \begin{align*}
        \text{det} \left( \overline{f}'(s) , \overline{f}''(s) \right) &= \text{det} \left( T(s),
        k(s)N(s)\right) \\
        &= k(s)
    \end{align*}
    since, 
    \[
        f''(s) = k(s)N(s) 
    \]
    \[
        k(s) \coloneqq \langle f''(s) , N(s) \rangle = \pm \| f''(s) \|^{ }_{ } 
    \]
\end{proof}

\begin{figure}[ht]
    \centering
    \incfig{curvature}
    \caption{curvature}
    \label{fig:curvature}
\end{figure}








\subsection{Osculating Circle and Center of Curvature}
\label{subsec:Osculating Circle and Center of Curvature}
\begin{defn}[]
    \[
        c(t) = f(t) + \frac{ 1 }{ k(t) } N(t) 
    \]
    is called the center of curvature. 
    \[
        \frac{ 1 }{ \left | k(t) \right |  } 
    \]
    is the radius of curvature at $ f(t) $
    The circle 
    \[
        \mathscr{ C } \left( c(t), \frac{ 1 }{ \left | k(t) \right |  } \right) 
    \]
    The evolute of $ f $ is the set of centers of curvatures.
\end{defn}


\subsection{Serret-Fresnet Formula}
\label{subsec:Serret-Fresnet Formula}
\begin{prop}[]
    \[
        T'(s) = k(s)N(s) \qquad N'(s) = -k(s)T(s)  
    \]
    lhs done is done, and rhs is by definition 
\end{prop}

\subsection{Total Curvature}
\label{subsec:Total Curvature}

\begin{ftheo}[Total Curvature]
    Let $ f: I \subset \mathbb{R}\to \mathbb{R}^2 $ planar curve parametrized by
    arc-length, then 
    \[
        \int\limits_{a}^{b} k(s) \ ds = \theta(a,b)
    \]
    is the angle between the two tangents at a and b.
    \label{th:Total Curvature}
\end{ftheo}


\begin{proof}
    \[
        f(s) = \begin{pmatrix*}
            x(s)  \\
            y(s)  
        \end{pmatrix*}
        \quad \theta (s) = \left( (o,x) , T(s) \right) 
    \]
    where $ o  $ is the angle between tangent and the x? 
    Then 
    \[
        T(s) = \begin{pmatrix*}
            \cos \theta (s)   \\
            \sin \theta (s)   \\
        \end{pmatrix*}
        \quad 
        N(s) = 
        \begin{pmatrix*}
            -\sin \theta (s)  \\
            \cos \theta (s)   
        \end{pmatrix*}
        
    \]
    However, 
    \[
        T'(s) = k(s) N(s) \quad \text{ and } T'(s) = \theta '(s) N(s) 
    \]
    then 
    $ \theta '(s) = k(s)  $. 
    then 
    \[
        \int\limits_{a}^{b} \theta '(s) ds = \theta(b) - \theta(a) 
    \]
    which is the difference between the angles. 
\end{proof}

\begin{figure}[ht]
    \centering
    \incfig{total-curvature}
    \caption{Total Curvature}
    \label{fig:total-curvature}
\end{figure}

In figure \ref{fig:total-curvature} we have $ \theta_1\left( a,b\right) = \theta_2\left(
a,b\right) = \theta_3\left( a,b\right) + 6\pi $. We defined the "winding number" as an
index refering to the full revolutions around the a curve that is closed of class $
\mathscr{ C } ^2 $.  

\[
    k = \frac{ l }{ 2\pi  } \int\limits_{a}^{b} k(s) \ ds \in \mathbb{Z}
\]


\begin{figure}[ht]
    \centering
    \incfig{winding-numbers}
    \caption{Winding Numbers}
    \label{fig:winding-numbers}
\end{figure}

\subsubsection{Concluding Thoughts}
\begin{itemize}
  \item Metrics of a curve are given by the 1st derivative
  \item Them shape of a curve is given by the second derivative. 
  \item Arc length parametrization gives constant speed along the curve 
\end{itemize}






