\chapter{Space Curves} 
\section{Definition}
\label{sec:Definition}
\begin{defn}[Space Curve]
    A space curve is a curve in $ \mathbb{R}^3 $ which is not planar. 
    \[
    f: I \subset \mathbb{R}^2 \to \mathbb{R}^3 \ \mathscr{ C } ^k \ k \geq 2
    \]
    \label{def:Space Curve}
\end{defn}



\begin{figure}[ht]
    \centering
    \incfig{space-curves}
    \caption{Space Curve}
    \label{fig:space-curves}
\end{figure}





Length is calculated the exact same way. 
\section{Curvature and Principle Normal}
\label{sec:Curvature and Principle Normal}
\begin{defn}[]
    if $ f : I \to \mathbb{R}^3 $ is arc-length param then the curvature is defined by 
    \[
        k(s) = \| f''(s) \|^{ }_{ } 
    \]
    NOTE: 
    $ \\ $
    $ k(s) \geq 0 $ for $ \mathbb{R}^3 $ 
    but in $ \mathbb{R}^2, k(s) = \langle f'(s)  , N(s)  \rangle = \pm \|
    f''(s)\|^{ }_{ }   $ 
    \label{def:}
\end{defn}

\begin{prop}
    If $ f $ is any parametrization 
    \[
        k(t) = \frac{ \| f'(t) \wedge f''(t)  \|^{ }_{ }  }{ \| f'(t) \|^{ 3}_{ }  } 
    \]
\end{prop}

\begin{proof}
    We denote $ \bar{f} = f \circ \sigma^{-1}  $ the arc-length parametrization. 
    \[
        \| \bar{f}' (s)  \|^{2 }_{ } = 1 \implies 2 \langle \bar{f}''(s) , \bar{f}'(s)
        \rangle = 0 
    \]
    \[
        \implies \bar{f}''(s) \perp \bar{f}'(s) 
    \]
    \[
        \imples k(s) = \| \bar{f}''(s) \|^{ }_{ } = \| \bar{f}'(s) \wedge \bar{f}''(s)  \|^{ }_{ } 
    \]
    and $ \bar{f} = f \circ \sigma^{-1}  $
    \[
        \implies \bar{f}'(s) = f'\left(\overbrace{\sigma^{-1}(s)}^{= t}  \right) \times
        \left( \sigma^{-1}\right) '(s) = \frac{ f'(t) }{ \| f'(t) \|^{ }_{ }  } 
    \] 
    \[
        \bar{f}''(s) = f''\left( \overbrace{\sigma^{-1}(s)}^{=t}\right) \times \left(
        \sigma^{-1} ' (s) \right) ^2 + f'\left( \sigma^{-1}(s) \right) \times \lambda
        \quad \lambda \in \mathbb{R}
    \]
    then 
    \[
        \| \bar{f}'(s) \wedge \bar{f}''(s)  \|^{ }_{ } = \| \frac{ f'(t) }{ \| f'(t) \|^{
        }_{ }} \wedge \frac{ f''(t) }{ \| f'(t) \|^{ 2}_{ }  }   \|^{ }_{ } 
    \]
\end{proof}


\begin{defn}[]
    Suppose $ f: I \to \mathbb{R}^3 $ is arc-length then 
    \[
        \vec{N}(s) = \frac{ 1 }{ k(s)  } \vec{T}(s) = \frac{ f''(s)  }{ \| f''(s) \|^{ }_{ }  }     
    \]
    is called the principle normal and 
    \[
        \vec{T}(s) = f'(s) \text{ is tangent at } f(s)
    \]
    a point $ f(s)  $ is biregular if $ f''(s) \neq 0 $
    \label{def:}
\end{defn}
\subsubsection{Remark}
A point is biregular if the curvature is $ \neq  $ from 0. 

\begin{figure}[ht]
    \centering
    \incfig{nt-for-r3}
    \caption{NT for R3}
    \label{fig:nt-for-r3}
\end{figure}




\begin{defn}[Osculating Plane]
    The osculating plane at a biregular point is spanned by $ T(s) $ and $ N(s) $
    \label{def:Osculating Plane}
    \begin{itemize}
        \item $ R(s)  = \frac{ 1 }{ k(s) } $ is the readius of curvature 
        \item $ C(s) = f(s) + R(s)\vec{N}(s)  $ center of curvature 
        \item $ \mathscr{ S } \left( C(s), R(s) \right)  $ osculating sphere 
        \item $ \mathscr{ S } \left( C(s), R(s) \right) \cap \left( \text{Osculating
            Plane} \right)  $ gives us the soculating circle
    \end{itemize} 
\end{defn}

\section{Serret-Frenet frame}
\label{sec:Serret-Frenet frame}
\begin{defn}[Serret-Fresnet Frame]
    if $ f $ is defined by arc-length biregular then 
    \[
        B(s) = T(s) \wedge N(s) 
    \]
    is called the binormal, and 
    \[
        \left( f(s), T(s), N(s) , B(s) \right) 
    \] is called the Serret-Fresnet 
    \label{def:Serret-Fresnet Frame}
\end{defn}
\subsubsection{Remark}
This frame does not exist at non-biregular points

\begin{defn}[]
    \begin{itemize}
        \item the plane : $ \langle N(s) , T(s)  \rangle  $ osculating plane 
        \item the plane : $ \langle N(s) , B(s)  \rangle  $ normal plane 
        \item the plane : $ \langle T(s) , B(s)  \rangle  $ rectifiable plane 
    \end{itemize} 
    \label{def:}
\end{defn}

\section{Tortion }
\label{sec:Tortion }
\begin{prop}
    $ B'(s)  $ is colinear to $ N(s) $
\end{prop}
\begin{proof}
\end{proof}

\begin{defn}[Torsion]
    At a biregular point, $ f\in \mathscr{ C } ^3 $ the torsion $ \tau (s)  $ is defined by 
    \[
        B'(s) = -\tau (s) N(s)
    \]
    It is a "measure" of how the osculating plane varies and twists 
    \label{def:Torsion}
\end{defn}

\begin{prop}
    \[
        \tau(s) = \frac{ \det \left( f'(s), f''(s), f'''(s) \right)  }{ \| f''(s) \|^{ 2}_{ }  }   
    \]
\end{prop}
\begin{proof}
    \[
        \tau(s) = - \langle N(s) , B'(s) \rangle  \quad B(s) = T(s) \wedge N(s)
    \]
    \[
        \implies B'(s) = T'(s) \wedge N(s) + T(s) \wedge N'(s) 
    \]
    $ \implies  $
    \[
        \tau(s) = \underbrace{ - \langle N(s) , T'(s) \wedge N(s) \rangle}_{=0} - \langle
        N(s) , T(s) \wedge N'(s)  \rangle 
    \] by definition 
    \[
        = \det\left( N(s), T(s), N'(s)\right) 
    \]
    However, 
    $ N(s) = R(s)f''(s)  $ and $ T(s) = f'(s) $ which gives 
    \[
        N'(s) = R'(s) f''(s) + R(s) f'''(s) 
    \]
    \[
        \tau(s) = \det \left( R(s) f''(s), f'(s) , \underbace{R'(s)f''(s)}_{\text{colinear
        to} R(s)f''(s) } + R(s)f'''(s)\right) 
    \]
    so we can invert the two vectors and introduce a minus sign to get 
    \[
        = \det\left( f'(s), f''(s), f'''(s)\right) R^2(s)
    \]
    and 
    \[
        R(s) = \frac{ 1 }{ k(s) } = \frac{ 1 }{ \| f''(s) \|^{ }_{ }  } 
    \]
\end{proof}

\begin{prop}
    For any param for biregular points we have 
    \[
        \tau(t) = \frac{ \det \left( f'(t), f''(t), f'''(t)\right)  }{ \| f'(t) \wedge
        f''(t) \|^{ 2}_{ }  } 
    \]
\end{prop}
\begin{proof}
    Admitted
\end{proof}

\begin{prop}
    $ f \in \mathscr{ C } ^3 $ biregular $ f : I \to \mathbb{R}^3 $
    $ \tau(s) \equiv 0 \iff f \text{ planar }  $
\end{prop} 
\begin{proof}
    $ \leftarrow   $ $ T(s)  $ and $ N(s) $ are in the same plane then $ B(s) $ constant
    and $ B'(s) = 0  $ then $ \tau \equiv 0 $. 
    $ \\ $
    $ \implies  $ if $ \tau \equiv 0  $ and $ B'(s) = - \tau(s) N(s)  $ \implies $ B'(s) =
    0 \forall s $ $ \implies B(s) = \vec{B}_0 $ 
    we show that 
    $ \langle B_0 , f(s) \rangle  = $ constant.  
    $ \\ $However, 
    \[
        \left( \langle B_0 , f \rangle \right) ' = \langle B_0 ,  \rangle 
    \]
\end{proof}

\section{Serret-Fresnet Formla}
\label{sec:Serret-Fresnet Formla}
\begin{defn}[SF Formula ]
    \[
    \begin{cases}
        T'(s) &= k(s)N(s) \\
        N'(s)  &= k(s)T(s) + \tau(s)B(s) \\ 
        B'(s) &= -\tau(s)N(s) 
    \end{cases}
    \]
    \label{def:SF Formula }
\end{defn}
\begin{proof}
    1,3 ok. 

\end{proof}

\section{Fundamental Theorem for Local Theory of Curves}
\label{sec:Fundamental Theorem for Local Theory of Curves}
\begin{ftheo}[]
    Let $ k :[a,b] \to \mathbb{R}^t  \in \mathscr{ C } ^1 $ and $ \tau : [a,b] \to
    \mathbb{R} \in \mathscr{ C } ^1$ then $ \exists ! $ curve $ f:[a,b] \to \mathbb{R}^3
    \in \mathscr{ C } ^3  $ parameterized by arc-length with curvature $ k $ and torsion
    $ \tau  $ up to a rigid transformation. 
    \label{th:}
\end{ftheo}
So a curve is completely determined by its tortion and curvature. 


\section{Tutorial 3: Plane and Space Curves}
\label{sec:Tutorial 3: Plane and Space Curves}
\subsubsection{Exercise 1}
Calculate the curvature of the planar curve parametrized by $ f(t) = \left( t, \varphi(t) \right)  $

\subsubsection{Exercise 2}
\begin{enumerate}
    \item Give a parametrization of the circle of center $ O  $ and radius $ R  $. 
    \item Calculate the curvature, radius of curvature and center of curvature at every
        point of the circle. 
\end{enumerate}

\subsubsection{Exercise 3}
Show that planar curves with constant curvature are either arc of circles or segments of a
line. 

\subsubsection{Exercise 4 (Circular Helix) }
We consider the curve $ f : \mathbb{R} \to \mathbb{R}^3 $ parameterized by 
\[
    f(t) = \left( R\cos t, R \sin t, at \right) 
\]
\begin{enumerate}
    \item Determine an arc-length representation
    \item Dtermine the Serret-Fresnet frame
    \item Calculate the curvature and torsion
    \item Calculate the set of curvature centers 
\end{enumerate}

\subsubsection{Exercise 5}
Calculate the evolute (i.e. the set of the centers of curvature) of the ellipse. 

\subsubsection{Exercise 6}
We consider the curve $ \gamma(t) = \left( t^2, t^3\right)   $ for $ t\geq 0 $
\begin{enumerate}
    \item Draw the curve. Is the curve regular? 
    \item Calculate the arc-length, curvature, and express the curvature with arc-length
        parametrization for $ t > 0 $. 
    \item Provide a parametrixation of the curve of class $ \mathscr{ C } ^1  $ and
        regular. 
    \item Show that there is no parametrization of class $ \mathscr{ C } ^2 $ of the curve
        at the point $ \left( 0,0\right)  $
\end{enumerate}
\subsection{Solutions}
\label{subsec:Solutions}

